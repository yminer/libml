\part{The \textit{mld} daemon}
\label{mld}

\section{The Machine Learning Daemon}

\subsection{Introduction}

The Machine Learning Library can be used in two manners :
\begin{itemize}
\item As a classic library linked with the main program.
\item As a daemon.
\end{itemize}

Why a daemon? The aim is the interoperability and the quick development 
of software in many languages. The interoperability is ensured by SOAP bindings, 
a little introduction to SOAP will be made in \vref{mld:soap}.\\

\subsection{\emph{mld} and SOAP}
\label{mld:soap}

SOAP stands for \textbf{Simple Object Access Protocol}. It is a simple XML 
based protocol to let applications easily exchange information 
over HTTP.\\
Quickly, the strenghts of the SOAP protocol :
\begin{enumerate}
\item platform independant
\item language independant (more than 60 languages offers SOAP support)
\item based on XML
\item developed as W3C standard
\end{enumerate}

Why SOAP? One of the greatest problem in library design is the choice of the languages that
will be handled. The most part of today's libraries are developped for one language (the language
in which they are implemented).\\
In the Machine Learning Library developpment, SOAP give us an abstract API, which can be used in
all languages with a SOAP support.\\

Finaly, SOAP provides an efficient way to communicate between applications running on different operating
systems, with different technologies and programming languages.

\subsubsection{The way of grid computing}

SOAP let us consider grid computing architectures design. Indeed the facilities provided by 
SOAP can be used in order to implement distributed machine learning stuffs.

\subsection{Using \emph{mld}}

LibML's compilation system uses the gSOAP project to generate
a \textit{libml.wsdl} which contains the definitions of the functions provided by \emph{mld}.\\

\noindent Using \emph{mld} is quite easy :
\begin{itemize}
\item launch \emph{mld} with this command :\\
\begin{verbatim}
matthieu@gally > mld <host-name> <port>
\end{verbatim}
\item compile your program using the soap library generated from the file
\emph{libml.wsdl}.
\item launch it!
\end{itemize}

Not all languages provides a SOAP handling with an automatic creation of the SOAP bindings 
from the wsdl file. If it is your case you can write your own bindings (its requiring more work), 
read the documentation of the SOAP library for this language.


\newpage
\section{Using \textit{mld} in C/C++}

\subsection{Generating client library with gSOAP from wsdl}

\noindent The gSOAP project provides two applications :
\begin{description}
\item [soapcpp2 :] This program take the libml.h and generate the server source code. This program
is used to generate \emph{mld}.
\item [wsdl2h :] This program takes the libml.wsdl and generate the client source code.\\
\end{description}

\noindent In order to generate C bindings for \emph{mld} :
\begin{verbatim}
matthieu@gally > wsdl2h -c libml.wsdl
\end{verbatim}

\noindent This command generate the files \textbf{soapC.c} and \textbf{soapClient.c} and the headers.\\

\noindent There is two manners to compile your application with this files.

\subsubsection{1 - You have installed the package gSOAP and its libraries}

\noindent So the compilation command is :
\begin{verbatim}
matthieu@gally > gcc -o my-soft soapC.c soapClient.c my-soft.c -lgsoap
\end{verbatim}

\subsubsection{1 - You have not installed the package gSOAP and its libraries}

\noindent You don't have to install this package to compile your application. You just have
to take the files \textbf{stdsoap2.c} and \textbf{stdsoap2.h} and copy it in your working directory.
\begin{verbatim}
matthieu@gally > gcc -o my-soft soapC.c soapClient.c stdsoap2.c my-soft.c
\end{verbatim}

The generation of C++ bindings uses the same rules, you just have to use this command :
\begin{verbatim}
matthieu@gally > wsdl2h libml.wsdl
\end{verbatim}
This command imply the generation of C++ files instead of C files.\\

That's all!

\subsection{Code example}
A code example can be found in the distribution tarball. look at
\textit{doc/examples/c}.

\newpage
\section{Using \textit{mld} in Perl}

\newpage
\section{Using \textit{mld} in Java}

The following resources are available for developers willing to make a Java
program which uses LibML:
\begin{itemize}
\item A \textit{libml-java-api-x.x.jar} file located in \textit{stubs/java} which contains all
the needed classes to connect to mld and make a learning.
\item The \textit{LibML Java API documentation}, a javadoc-like documentation
located in the jar file, and in the main documentation page of the
distribution tarball (see \textit{doc/index.html}).
\item An example of the way the \textit{jar} file should be used, located in
\textit{doc/examples/java}.
\end{itemize}

\subsection{Requirements}

\begin{itemize}
\item A compiled \textit{mld} binary.
\item The \textit{libml-java-api-x.x.jar} file provided in the distribution
tarball.
\item The \textit{jar} files of the \emph{Apache Axis}\footnote{Apache Axis:
an open source implementation of the SOAP ("Simple Object Access Protocol",
\url{http://ws.apache.org/axis/}) submission to W3C.} library.
\item A java compiler. The example uses \emph{gcj}\footnote{gcj: the GNU
compiler for Java, \url{http://gcc.gnu.org/java/}}.
\item A Java virtual machine. The example uses \emph{kaffe}\footnote{kaffe: a
GPL Java virtual machine, \url{http://www.kaffe.org/}} but the jar file
should be usable with other VM's, and obviously Sun's one.
\end{itemize}

\subsection{Using the provided \textit{jar} file}
Basically, you need to create a class which inherits from the
\textit{org.libml.driver.LibmlDriver} virtual class. This way, you get an
object which is able to connect to a running instance of \textit{mld} thanks
to the inherited \textit{connect} method, and you just have to implement the
\textit{run} method which actually sends the instructions which make the
learning you need.

\subsection{Code example}
Take a look at the example's sources and the
\textit{LibML Java API documentation} to understand what's going on. Reading
the Makefile should help understanding how the code is built and run.

\subsubsection{Requirements}
\begin{itemize}
\item In order to build the example, you need gcj
\item In order to run them, you need kaffe.
\end{itemize}
Note: you can use another compiler/VM with a little hack in the Makefile.

\subsubsection{Building the example}
\begin{verbatim}
olivier@freedom libml-x.x > cd doc/examples/java
olivier@freedom java > make
\end{verbatim}

\subsubsection{Running the example}
First, you need to run mld in another terminal so that it listens on the
port 4242 of 127.0.0.1 (localhost):
\begin{verbatim}
olivier@freedom libml-x.x > cd src/server
olivier@freedom server > ./mld ``127.0.0.1'' 4242
\end{verbatim}
Then, run the example in the first terminal:
\begin{verbatim}
olivier@freedom java > make run
\end{verbatim}
The executed Java code is supposed to connect to \textit{mld} and send dummy
instructions to it.

\newpage
\section{Using \textit{mld} in Python}

\newpage
\section{Using \textit{mld} in C\#}
