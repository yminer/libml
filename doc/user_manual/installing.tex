\part{Installing}
\label{installing:part}

\chapter{Requirements}
\section{Software dependencies}
\label{installing:deps}
The following softwares have to be installed on your system if you want to
build, install, or use LibML:
\begin{itemize}
\item{\textbf{The OCaml distribution} (`apt-get install ocaml'). Sources available
here: http://caml.inria.fr/ocaml/distrib.html}
\item{\textbf{The `findlib' library} (`apt-get install ocaml-findlib'). Sources
available here: http://ocaml-programming.de/programming/findlib.html}
\item{\textbf{gcc} (`apt-get install gcc'). Used to build the mld daemon.}
\end{itemize}

%TODO url mailing list
\section{Supported architectures/operating systems}
LibML is usually developped on \gloss{GNULINUX} systems. Please report any compilation problem on our development mailing list (\url{http://??}\cite{libml:mailinglist}).


\chapter{Installing LibML}
\label{installing:packaging}
\section{Getting the source code}
LibML's source code is available on \url{http://libml.org}\cite{libml:web}.
It is distributed as tarballs compressed using bzip2, named libml-x.x.tar.bz2.
\section{Decompressing the tarball}
The following command will decompress the tarball and dump its content in a
new subdirectory named libml-x.x:
\begin{verbatim}
olivier@freedom ~ > bzip2 -dc libml-x.x.tar.bz2 | tar xv
\end{verbatim}
Here is what one can find in the tarball:
\begin{itemize}
\item LibML's sources, of course
\item a set of documentations. See \vref{installing:documentation}.
\end{itemize}

\section{Compiling the library}
\label{installing:compilation}
The following command will compile LibML\footnote{You need
\textit{GNU make}.}:
\begin{verbatim}
olivier@freedom libml-x.x > make
\end{verbatim}
\subsection*{Compilation problems}
If you have any problem compiling, please check these
\begin{enumerate}
\item If compilation failed to begin because of an error while the
\textit{src/configure} script was being run, check the output. Problem
at this step are probably due to a problem with the dependencies (see \vref{installing:deps}).
\item Check that you are using \textit{GNU make}.
\end{enumerate}
Please report compilation problems on our mailing list \cite{libml:mailinglist}.

\section{Installing LibML}
The following command will install the library you've just compiled and
the documentation on your system:
\begin{verbatim}
olivier@freedom libml-x.x > make install
\end{verbatim}

\section{Documentation}
\label{installing:documentation}
%TODO find a way to get a real link here
The \textit{doc/index.html} file enables you to access all the documentation
located in \textit{doc/} using your favorite web browser.
\section{Uninstalling LibML}
The following command will uninstall LibML and its documentation from
your system:
\begin{verbatim}
olivier@freedom libml-x.x > make uninstall
\end{verbatim}
